\documentclass{article}

\usepackage{amsmath}
\usepackage{booktabs}
\usepackage{bytefield}
\usepackage{caption}
\usepackage{endnotes}
\usepackage{fancyvrb}
\usepackage{float}
\usepackage{longtable}
\usepackage{minibox}
\usepackage{register}
\usepackage{standalone}
\usepackage{swiftnav}
\usepackage{tabularx}
\usepackage{tocloft}
\usepackage{setspace}
\usepackage{pbox}
\usepackage{soul}
\usepackage{hyperref}

\hypersetup{bookmarks,bookmarksopen,bookmarksdepth=4}

\setlength{\regWidth}{0.4\textwidth}

\floatstyle{plain}
\newfloat{field}{h}{fld}
\floatname{field}{Field}

\numberwithin{table}{subsection}
\numberwithin{field}{subsection}

\renewcommand{\thesubsubsection}{\hspace{-0.45cm}}
%% \renewcommand{\thesubsection}{\arabic{section}.\arabic{subsection}}
%% \makeatletter
%% \def\@seccntformat#1{\csname #1ignore\expandafter\endcsname\csname the#1\endcsname\quad}
%% \let\sectionignore\@gobbletwo
%% \let\latex@numberline\numberline
%% \def\numberline#1{\if\relax#1\relax\else\latex@numberline{#1}\fi}
%% \makeatother


\renewcommand{\regLabelFamily}{}

\cftsetindents{section}{0.5in}{0.5in}
\cftsetindents{subsection}{0.5in}{0.5in}
%%\setlength\cftparskip{-1.2pt}
\setlength{\cftbeforetoctitleskip}{-1em}
\setlength\cftbeforesecskip{1.3pt}
\setlength\cftaftertoctitleskip{2pt}
\renewcommand{\cftsecafterpnum}{\hspace*{7.5em}}
\renewcommand{\cftsubsecafterpnum}{\hspace*{7.5em}}
\renewcommand\tableofcontents{\@starttoc{toc}}

\newcolumntype{a}{>{\hsize=.2\hsize}X}
\newcolumntype{b}{>{\hsize=.4\hsize}X}
\newcolumntype{c}{>{\hsize=.3\hsize}X}
\newcolumntype{d}{>{\hsize=.7\hsize}X}

\title{SwiftNav Binary Protocol}
\version{0.15}
\author{Swift Navigation}
\mysubtitle{Protocol Specification v\theversion}
\date{\today}

\begin{document}
\maketitle
\begin{normalsize}
\setcounter{tocdepth}{2}
\begin{centering}
\tableofcontents
\end{centering}
\end{normalsize}

\thispagestyle{firstpage}
\bigskip
\bigskip
\begin{large}
The Swift Navigation Binary Protocol (SBP) is a fast, simple, and
minimal binary protocol for communicating with Swift devices. It is
the native binary protocol used by the Piksi GPS receiver to transmit
solutions, observations, status and debugging messages, as well as
receive messages from the host operating system, such as differential
corrections and the almanac. As such, it is an important piece of
interfacing with your Piksi receiver and integrating it with other
systems.

This document provides language-agnostic specification and
documentation for messages used with SBP. SBP client libraries in a
variety of programming languages are available at
\url{http://docs.swiftnav.com/wiki/SwiftNav_Binary_Protocol}.
\end{large}

\newpage
\section{Message Structure}
\label{sec:Message}

\begin{large}
SBP consists of two pieces: (i) an over-the-wire message framing
format and (ii) structured payload definitions. As of Version 1.0, the
packet consists of a 6-byte binary header section, a variable-sized
payload field, and a 16-bit CRC value. SBP uses the CCITT CRC16
(XMODEM implementation) for error detection.
\end{large}

\begin{table}[h]
  \centering
  \begin{tabularx}{\textwidth}{aaX}
    \toprule
    Name & Size & Description \\
    \midrule
    {Preamble} & $1$ & Denotes the start of frame transmission. Always 0x55. \\
    {Message Type} & $2$ & Identifies the payload contents. \\
    {Sender} & $2$ & \hangindent=0.5em{A unique identifier of the sending hardware. Set to the 2 least significant bytes of the Piksi serial number.} \\
    {Length} & $1$ & Length in bytes of the {Payload} field. \\
    {Payload} & $N$ & Binary data of the message. \\
    {CRC} & $2$ & \hangindent=0.5em{Cyclic Redundancy Check of the packet's binary data from the Message Type up to the end of Payload (does not include the Preamble).} \\
    \midrule
    & $N+8$ \\
    \bottomrule
  \end{tabularx}
  \caption{Swift Binary Protocol message structure}
  \label{tab:message}
\end{table}
\newpage

\section{Basic Formats and Payload Structure}
\label{sec:Payload}
\begin{large}
The binary payload of an SBP message can be decoded into structured
data based on the message type defined in the header. SBP uses several
primitive numerical and collection types for defining the contents of
these payloads:
\end{large}
\begin{table}[h]
  \centering
  \begin{tabularx}{0.80\textwidth}{aaX}
    \toprule
    Name & Size & Description \\
    \midrule
    ((*- for t in prims *))
    (((t.identifier))) & (((t.identifier | getsize))) & \hangindent=0.5em{(((t.desc)))} \\
    ((*- endfor *))
    \bottomrule
  \end{tabularx}
  \caption{SBP primitive types}
  \label{tab:types}
\end{table}
\begin{large}
 \par As an example, consider this deframed series of bytes read
from a serial port:
\begin{verbatim}
55 02 02 cc 04 14 70 3d d0 18 cf ef ff ff ef e8 ff ff f0 18 00 00 00 00 05 00 43 94
\end{verbatim}
This cryptic byte array decodes into a \texttt{MSG\_BASELINE\_ECEF}
(see pg.~\pageref{sec:MSG_POS_ECEF}), which reports the baseline
position solution of the rover receiver relative to the base station
receiver in Earth Centered Earth Fixed (ECEF) coordinates. The
segments of this byte array and its contents breakdown as follows:
\end{large}
\begin{table}[h]
  \centering
  \begin{tabular}{llrl}
    \toprule
    Field Name & Type & Value & Bytestring Segment\\
    \midrule
    {Preamble} & u8 & 0x55 & \verb|55| \\
    {Message Type}& u16 & 0x0202 & \verb|02 02| \\
    {Sender}& u16 & 0x4cc & \verb!cc 04! \\
    {Length}& u8 & 20 &  \verb!14! \\
    {Payload}& & --- & \verb!70 3d d0 18 cf ef ff ff ef e8 ff ff f0 18 00 00 00 00 05 00! \\
    \quad~\texttt{MSG\_BASELINE\_ECEF} & & & \\
    \quad~.tow & u32 & $416300400~\textrm{sec}$  & \verb!70 3d d0 18! \\
    \quad~.x & s32 & $-4145~\textrm{mm}$  & \verb!cf ef ff ff! \\
    \quad~.y & s32 & $-5905~\textrm{mm}$  & \verb!ef e8 ff ff! \\
    \quad~.z & s32 & $6384~\textrm{mm}$  & \verb!f0 18 00 00! \\
    \quad~.accuracy & u16 & 0 & \verb!00 00! \\
    \quad~.nsats & u8 & 5 & \verb!05! \\
    \quad~.flags & u8 & 0 & \verb!00! \\
    {CRC} & u16 & 0x9443 & \verb!43 94! \\
    \bottomrule
  \end{tabular}
  \caption{SBP breakdown for \texttt{MSG\_BASELINE\_ECEF}}
  \label{tab:example_breakdown}
\end{table}

((* block messages_table *))
((* endblock *))

((* block messages_desc *))
((* endblock *))

\end{document}
