\documentclass{article}

\usepackage{amsmath}
\usepackage{booktabs}
\usepackage{bytefield}
\usepackage{caption}
\usepackage{endnotes}
\usepackage{fancyvrb}
\usepackage{float}
\usepackage{longtable}
\usepackage{minibox}
\usepackage{register}
\usepackage{standalone}
\usepackage{swiftnav}
\usepackage{tabularx}
\usepackage{tocloft}
\usepackage{setspace}
\usepackage{pbox}
\usepackage{soul}
\usepackage{hyperref}

\hypersetup{bookmarks,bookmarksopen,bookmarksdepth=4}

\setlength{\regWidth}{0.4\textwidth}

\floatstyle{plain}
\newfloat{field}{h}{fld}
\floatname{field}{Field}

\numberwithin{table}{subsection}
\numberwithin{field}{subsection}

\renewcommand{\thesubsubsection}{\hspace{-0.45cm}}
%% \renewcommand{\thesubsection}{\arabic{section}.\arabic{subsection}}
%% \makeatletter
%% \def\@seccntformat#1{\csname #1ignore\expandafter\endcsname\csname the#1\endcsname\quad}
%% \let\sectionignore\@gobbletwo
%% \let\latex@numberline\numberline
%% \def\numberline#1{\if\relax#1\relax\else\latex@numberline{#1}\fi}
%% \makeatother

\newcommand{\specialcell}[2][c]{%
  \begin{tabular}[#1]{@{}c@{}}#2\end{tabular}}

\renewcommand{\regLabelFamily}{}

\cftsetindents{section}{0.5in}{0.5in}
\cftsetindents{subsection}{0.5in}{0.5in}
%%\setlength\cftparskip{-1.2pt}
\setlength{\cftbeforetoctitleskip}{-1em}
\setlength\cftbeforesecskip{1.3pt}
\setlength\cftaftertoctitleskip{2pt}
\renewcommand{\cftsecafterpnum}{\hspace*{7.5em}}
\renewcommand{\cftsubsecafterpnum}{\hspace*{7.5em}}
\renewcommand\tableofcontents{\@starttoc{toc}}

\newcolumntype{a}{>{\hsize=.2\hsize}X}
\newcolumntype{b}{>{\hsize=.4\hsize}X}
\newcolumntype{c}{>{\hsize=.3\hsize}X}
\newcolumntype{d}{>{\hsize=.7\hsize}X}

\title{Swift Navigation Binary Protocol}
\version{0.33}
\author{Swift Navigation}
\mysubtitle{Protocol Specification v\theversion}
\date{\today}

\begin{document}
\maketitle
\begin{normalsize}
\setcounter{tocdepth}{2}
\begin{centering}
\tableofcontents
\end{centering}
\end{normalsize}

\thispagestyle{firstpage}
\bigskip
\bigskip
\begin{large}
\section{Overview}
\label{sec:Overview}
The Swift Navigation Binary Protocol (SBP) is a fast, simple, and
minimal binary protocol for communicating with Swift devices. It is
the native binary protocol used by the Piksi GPS receiver to transmit
solutions, observations, status, and debugging messages, as well as
receive messages from the host operating system, such as differential
corrections and the almanac. As such, it is an important interface
with your Piksi receiver and the primary integration method with other
systems.

This document provides a specification of SBP framing and the payload
structures of the messages currently used with Swift devices. SBP
client libraries in a variety of programming languages are available
at~\url{http://docs.swiftnav.com/wiki/SwiftNav_Binary_Protocol}.

\end{large}

\newpage
\section{Message Framing Structure}
\label{sec:Message}

\begin{large}
SBP consists of two pieces:
\begin{itemize}
  \item an over-the-wire message framing format
  \item structured payload definitions
\end{itemize}
As of Version~\theversion, the frame consists of a 6-byte binary
header section, a variable-sized payload field, and a 16-bit CRC
value. All multibyte values are ordered in \textbf{little-endian}
format. SBP uses the CCITT CRC16 (XMODEM implementation) for error
detection\footnote{CCITT 16-bit CRC Implementation uses parameters
  used by XMODEM, i.e. the polynomial: $x^{16} + x^{12} + x^5 +
  1$. For more details, please see the implementation
  at~\url{https://github.com/swift-nav/libsbp/blob/master/c/src/edc.c\#L59}. See
  also \emph{A Painless Guide to CRC Error Detection Algorithms}
  at~\url{http://www.ross.net/crc/download/crc_v3.txt}}.

\end{large}

\begin{table}[h]
  \centering
  \begin{tabularx}{\textwidth}{aaaX}
    \toprule
    Offset (bytes) & Size (bytes) & Name & Description \\
    \midrule
    $0$ & $1$ & {Preamble} & Denotes the start of frame transmission. Always 0x55. \\
    $1$ & $2$ & {Message Type} & Identifies the payload contents. \\
    $3$ & $2$ & {Sender} & \hangindent=0.5em{A unique identifier of the sender. On the Piksi, this is set to the 2 least significant bytes of the device serial number. A stream of SBP messages may also included sender IDs for forwarded messages.} \\
    $5$ & $1$ & {Length} & Length (bytes) of the {Payload} field. \\
    $6$ & $N$ & {Payload} & Binary message contents. \\
    $N+6$ & $2$ & {CRC} & \hangindent=0.5em{Cyclic Redundancy Check of the frame's binary data from the Message Type up to the end of Payload (does not include the Preamble).} \\
    \midrule
    & $N+8$ & &Total Frame Length \\
    \bottomrule
  \end{tabularx}
  \caption{Swift Binary Protocol message structure. $N$ denotes a variable-length size.}
  \label{tab:message}
\end{table}

\begin{large}
Swift devices, such as the Piksi, also support the standard NMEA-0183
protocol for single-point position solutions. Observations transmitted
via SBP can also be converted into RINEX. Note that NMEA doesn't
define a standardized message string for RTK solutions. To make it
possible to achieve RTK accuracy with legacy host hardware or software
that can only read NMEA, recent firmware versions implement a
``pseudo-absolute'' mode.

\end{large}

\newpage

\section{Basic Formats and Payload Structure}
\label{sec:Payload}
\begin{large}
The binary payload of an SBP message decodes into structured data
based on the message type defined in the header. SBP uses several
primitive numerical and collection types for defining payload
contents.
\end{large}
\begin{table}[h]
  \centering
  \begin{tabularx}{\textwidth}{aaX}
    \toprule
    Name & Size (bytes) & Description \\
    \midrule
    ((*- for t in prims *))
    (((t.identifier))) & (((t.identifier | getsize))) & \hangindent=0.5em{(((t.desc)))} \\
    ((*- endfor *))
    \bottomrule
  \end{tabularx}
  \caption{SBP primitive types}
  \label{tab:types}
\end{table}
\hspace{-5em}
\subsubsection*{Example Message}
\begin{large}
 \par As an example, consider this framed series of bytes read from a
 serial port:
\begin{verbatim}
55 02 02 cc 04 14 70 3d d0 18 cf ef ff ff ef e8 ff ff f0 18 00 00 00 00 05 00 43 94
\end{verbatim}
This byte array decodes into a \texttt{MSG\_BASELINE\_ECEF} (see
pg.~\pageref{sec:MSG_POS_ECEF}), that reports the baseline position
solution of the rover receiver relative to the base station receiver
in Earth Centered Earth Fixed (ECEF) coordinates. The segments of this
byte array and its contents breakdown as follows:
\end{large}
\begin{table}[h]
  \centering
  \begin{tabular}{llrl}
    \toprule
    Field Name & Type & Value & Bytestring Segment\\
    \midrule
    {Preamble} & u8 & 0x55 & \verb!55! \\
    {Message Type}& u16 & \texttt{MSG\_BASELINE\_ECEF} & \verb!02 02! \\
    {Sender}& u16 & 1228 & \verb!cc 04! \\
    {Length}& u8 & 20 &  \verb!14! \\
    {Payload}& & --- & \verb!70 3d d0 18 cf ef ff ff ef e8 ff ff! \\
    & & & \verb!f0 18 00 00 00 00 05 00! \\
    \quad~\texttt{MSG\_BASELINE\_ECEF} & & & \\
    \quad~.tow & u32 & $416300400~\textrm{msec}$  & \verb!70 3d d0 18! \\
    \quad~.x & s32 & $-4145~\textrm{mm}$  & \verb!cf ef ff ff! \\
    \quad~.y & s32 & $-5905~\textrm{mm}$  & \verb!ef e8 ff ff! \\
    \quad~.z & s32 & $6384~\textrm{mm}$  & \verb!f0 18 00 00! \\
    \quad~.accuracy & u16 & 0 & \verb!00 00! \\
    \quad~.nsats & u8 & 5 & \verb!05! \\
    \quad~.flags & u8 & 0 & \verb!00! \\
    {CRC} & u16 & 0x9443 & \verb!43 94! \\
    \bottomrule
  \end{tabular}
  \caption{SBP breakdown for \texttt{MSG\_BASELINE\_ECEF}}
  \label{tab:example_breakdown}
\end{table}

((* block messages_table *))
((* endblock *))

((* block messages_desc *))
((* endblock *))

\end{document}
