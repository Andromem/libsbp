\documentclass{article}

\usepackage{swiftnav}
\usepackage{bytefield}
\usepackage{endnotes}

\usepackage{standalone}

\usepackage{register}
\usepackage{bytefield}
\usepackage{booktabs}
\usepackage{minibox}
\usepackage{float}
\usepackage{amsmath}
\usepackage{caption}
\usepackage{tabularx}

\setlength{\regWidth}{0.4\textwidth}

\floatstyle{plain}
\newfloat{field}{h}{fld}
\floatname{field}{Field}

\numberwithin{table}{subsection}
\numberwithin{field}{subsection}

\usepackage{draftwatermark}
\SetWatermarkLightness{0.9}
\SetWatermarkScale{1}

\renewcommand{\regLabelFamily}{}

\newcolumntype{a}{>{\hsize=.2\hsize}X}
\newcolumntype{b}{>{\hsize=.4\hsize}X}
\newcolumntype{c}{>{\hsize=.3\hsize}X}
\newcolumntype{d}{>{\hsize=.7\hsize}X}

% ---------------------------------------------------------------------------

\version{1.0}
\title{SwiftNav Binary Protocol}
\mysubtitle{Protocol Specification v\theversion}
\author{Swift Navigation}
\date{\today}

\begin{document}

\maketitle

\thispagestyle{firstpage}

\section{Message Structure}
\label{sec:Message}


\begin{large}
The Swift Binary Protocol is a fast, simple and minimal binary
protocol for sending payloads to, from and between Swift-Nav
devices. It is primarily used to send the binary representation of C
structs with minimal overhead across serial links.

As of Version 1.0, the message consists of a 6 byte binary header
section, a variable-sized payload field, and a 2 byte binary CRC
value. SBP uses the CCITT CRC16 (XMODEM implementation) for error
detection. It has no error correction and makes no delivery
guarantees.
\end{large}

\begin{table}[h]
  \centering
  \begin{tabularx}{\textwidth}{aaX}
    \toprule
    Name & Size & Description \\
    \midrule
    \texttt{Preamble} & $1$ & Denotes the start of frame transmission. Always 0x55. \\
    \texttt{Message Type} & $2$ & Identifies the payload contents. \\
    \texttt{Sender} & $2$ & A unique identifier of the sending hardware. Set to the 2 least significant bytes of the Piksi serial number. \\
    \texttt{Length} & $1$ & Length in bytes of the \texttt{Payload} field. \\
    \texttt{Payload} & $N$ & Binary data of the message. \\
    \texttt{CRC} & $2$ & Cyclic Redundancy Check of the packet's binary data from the Message Type up to the end of Payload (does not include the Preamble). \\
    \midrule
    & $N+8$ \\
    \bottomrule
  \end{tabularx}
  \caption{Swift Binary Protocol message structure}
  \label{tab:message}
\end{table}

((* block messages_table *))
((* endblock *))

((* block messages_desc *))
((* endblock *))

\end{document}
